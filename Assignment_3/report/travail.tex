\typeout{------------------------------------------------------------------}
\typeout{} 
\typeout{        Fichier de base modifie par : Matth: 20 nov 2012} 
\typeout{                   sous licence GNU-GPL} 
\typeout{}
\typeout{------------------------------------------------------------------}

% Classe g�n�rale du document
   \documentclass[12pt]{report} % .10pt, 11pt, 12pt : taille de la police principale (10 par d�faut)
                 % .a4paper, letterpaper,... : d�limite la taille du papier. (letterpaper par d�faut)
	         % .fleqn : aligne les formules math�matiques � gauche au lieu de les centrer.
		 % .leqno : place la num�rotation des formules � gauche plut�t qu'� droite.
		 % .twocolumn : demande � LATEX de formater le texte sur deux colonnes.
		 % .twoside, oneside indique si la sortie se fera en recto-verso ou en recto simple.
		 % .landscape, mais il faut mettre en commentaire (ou modifier) toutes les dimmensions

% Importation de packages divers
%   \NeedsTeXFormat{LaTeX2e} 
   \usepackage[T1]{fontenc}
   \usepackage[utf8]{inputenc}		% utilisation des caract�res 8 bits en Unix (codage ISO 8859-1)
  %\usepackage[latin1]{inputenc}	% utilisation des caract�res pour Linux2
   \usepackage[usenames]{color}
   \usepackage{fancyhdr}
   \usepackage{lastpage}                % pour l'affichage du n� de la derni�re page.
   \usepackage{lmodern}
   \usepackage{multirow}                % pour l'utilisation de figures ``noy�es'' dans le texte
   \usepackage{xspace}			% package pour babel
   \usepackage[english]{babel}         % Utilisation du fran�ais (nom des sections, c�sure, ponctuation,...)
   \usepackage{amsmath,amsthm,amssymb}  % Utilisation de certains packages de AMS (cf. belles �quations)
   \usepackage{endnotes}                % Pour l'utilisation des notes en fin de documents
   \usepackage{verbatim}                % Pour l'insertion de fichier en mode verbatim
   \usepackage{portland}		% pour l'utilisation de \portrait et de \landscape sur une page
   \usepackage[pdftex]{graphicx}        % [pdftex] si utilisation d'images jgp,...
                                        % [dvips]  si utilisation d'images bmp,...
   \usepackage{pdfpages}		% Inclure des pages de pdf
   \usepackage{setspace}		% Pour d�finir un interligne
   \usepackage[bottom]{footmisc}        % Footnote at the bottom
%   \usepackage[cyr]{aeguill}		% Pour les guillemets � la Fran�aise
   \usepackage{eurosym}			% Pour les Euro
\usepackage{url}
\usepackage{tikz}
\usepackage{listings}
\usetikzlibrary{arrows,automata}

\urlstyle{sf}
	
   \renewcommand{\contentsname}{Content} % si tableofcontents au d�but
   \newcommand{\Numero}{\No}
   \newcommand{\numero}{\no}
%   \newcommand{\fup}[1]{\up{#1}}

   \DeclareGraphicsExtensions{.jpg,.pdf,.mps,.png}       % d�claration d'extensions  pour les images
   %\input xy                            % pour le package xy (construction de diagramme)
   %\xyoption{all}

% Dimensions de la page :       	

  %%%%%%%%%%%%%%%%%%%%%%%%%%%%%%%%%%%%%%%%  0
  %   |                                  %
  %---+----------------------------------%  1
  %   | +----------------------------+   %  2
  %   | |          en-t�te           |   %
  %   | +----------------------------+   %  3
  %   | +----------------------------+   %  4
  %   | |                            |   %       Remarques : 
  %   | |                            |   %        . distance de '0' � '1' : un pouce + \voffset
  %   | |                            |   %        . distance de 'a' � 'b' : un pouce + \hoffset
  %   | |           texte            |   %
  %   | |                            |   %
  %   | |                            |   %
  %   | |                            |   %
  %   | +----------------------------+   %  5
  %   | +----------------------------+   %
  %   | |         bas de page        |   %
  %   | +----------------------------+   %  6
  %%%%%%%%%%%%%%%%%%%%%%%%%%%%%%%%%%%%%%%%
  %a  b c                            d  e

%    % g�n�ral
%      \voffset       0mm    % pour descendre (si positif) ou remonter (si n�gatif) le tout
%      \hoffset       0mm    % pour agrandir (si positif) ou diminuer (si n�gatif) la marge gauche (distance 'a' 'b')
      \oddsidemargin 1mm   % 5pt  % distance de 'b' � 'c'
%     \evensidemargin 25mm  % 15pt % distance de 'd' � 'e'
%    % texte
%      \headsep       25pt   % distance de '3' � '4', la distance entre l'en-t�te et le texte
      \textheight    220mm  % distance de '4' � '5', pour d�terminer la hauteur du texte
      \textwidth     165mm  % distance de 'c' � 'd' 
%    % en-t�te
      \topmargin     0pt    % distance de '1' � '2', pour descendre (si positif) ou remonter (si n�gatif) le tout
      \headheight    15pt   % distance de '2' � '3', doit �tre > 14.49999
%    % bas de page
      \footskip      15mm   % 30pt % distance de '5' � '6', la distance entre le texte et le bas de page
     % space for the footnode
     \setlength{\skip\footins}{1cm}
     
% Mise en page
   \pagestyle{fancy}
%   \usepackage[Matth]{fncychap}

% (Re)d�finitions diverses

  % red�finition de l'affichage des titres de section dans l'en-t�te ou le bas de page
    % remarques :
    %  .affichage du num�ro (2)    : \thesection 
    %  .affichage du nom (Section) : \sectionname
    \renewcommand{\sectionmark}[1]{\markright{\thesection.\ #1}}   % 2.2. nom de la section 2.2
    \renewcommand{\thesection}{\arabic{section}}		% II nom de la section 0.2


  % des couleurs...                   (utilisation avec par ex. \textcolor{webdarkblue}{...})
   \definecolor{codeBlue}{rgb}{0,0,1}
   \definecolor{webred}{rgb}{0.5,0,0}
   \definecolor{codeGreen}{rgb}{0,0.5,0}
   \definecolor{codeGrey}{rgb}{0.6,0.6,0.6}
   \definecolor{webdarkblue}{rgb}{0,0,0.4}
   \definecolor{webgreen}{rgb}{0,0.3,0}
   \definecolor{webblue}{rgb}{0,0,0.8}
   \definecolor{orange}{rgb}{0.7,0.1,0.1}

  % utilisation de caption, label,... pour autre chose qu'une figure
        %%%% debut macro %%%%
   \makeatletter
   \def\captionof#1#2{{\def\@captype{#1}#2}}
   \makeatother
        %%%% fin macro %%%%


% remarques : 
%  . pour mettre la date                  : \today
%  . pour mettre le nom de la section     : \rightmark
%  . pour mettre le num�ro de page        : \thepage
%  . pour mettre le nombre de pages total : \pageref{LastPage}  (mais l'�crit en rouge vu que c'est une r�f.)
%  . insertion d'une image                : \setlength{\unitlength}{1mm}
%                                             \begin{picture}(0,0)
%                                                \put(5,0){\includegraphics[scale=x.x]{xxx.xxx}}
%                                             \end{picture}

% Pour les guillemets �  la Fran�aise
\newcommand{\fermerguillemets}{\unskip\kern.15em\symbol{20}}
\newcommand{\ouvrerguillemets}{\symbol{19}\ignorespaces\kern.15em}
\let �=\fermerguillemets
\let� =\ouvrerguillemets

% Pour changer l'icone des puces : � placer juste avant une liste
 %   \renewcommand\labelitemi{\textbullet}	% Style boulet :)
 
 
 
\definecolor{mygreen}{rgb}{0,0.6,0}
\definecolor{mygray}{rgb}{0.5,0.5,0.5}
\definecolor{mymauve}{rgb}{0.58,0,0.82}

\lstset{ %
  backgroundcolor=\color{white},   % choose the background color; you must add \usepackage{color} or \usepackage{xcolor}
%  basicstyle=\footnotesize,        % the size of the fonts that are used for the code
  basicstyle=\linespread{0.80}\ttfamily,
  breakatwhitespace=false,         % sets if automatic breaks should only happen at whitespace
  breaklines=true,                 % sets automatic line breaking
  captionpos=b,                    % sets the caption-position to bottom
  commentstyle=\color{mygreen},    % comment style
  deletekeywords={...},            % if you want to delete keywords from the given language
  escapeinside={\%*}{*)},          % if you want to add LaTeX within your code
  extendedchars=true,              % lets you use non-ASCII characters; for 8-bits encodings only, does not work with UTF-8
  frame=single,                    % adds a frame around the code
  keepspaces=true,                 % keeps spaces in text, useful for keeping indentation of code (possibly needs columns=flexible)
  keywordstyle=\color{blue},       % keyword style
  morekeywords={*,...},            % if you want to add more keywords to the set
%  numbers=left,                    % where to put the line-numbers; possible values are (none, left, right)
%  numbersep=5pt,                   % how far the line-numbers are from the code
 % numberstyle=\tiny\color{mygray}, % the style that is used for the line-numbers
  rulecolor=\color{black},         % if not set, the frame-color may be changed on line-breaks within not-black text (e.g. comments (green here))
  showspaces=false,                % show spaces everywhere adding particular underscores; it overrides 'showstringspaces'
  showstringspaces=false,          % underline spaces within strings only
  showtabs=false,                  % show tabs within strings adding particular underscores
  stepnumber=2,                    % the step between two line-numbers. If it's 1, each line will be numbered
  stringstyle=\color{mymauve},     % string literal style
  tabsize=2,                       % sets default tabsize to 2 spaces
%  title=\lstname                   % show the filename of files included with \lstinputlisting; also try caption instead of title
}
 
 \lstdefinelanguage{scala}{
  morekeywords={abstract,case,catch,class,def,%
    do,else,extends,false,final,finally,%
    for,if,implicit,import,match,mixin,%
    new,null,object,override,package,%
    private,protected,requires,return,sealed,%
    super,this,throw,trait,true,try,%
    type,val,var,while,with,yield},
  otherkeywords={=>,<-,<\%,<:,>:,\#,@},
  sensitive=true,
  morecomment=[l]{//},
  morecomment=[n]{/*}{*/},
  morestring=[b]",
  morestring=[b]',
  morestring=[b]"""
}


\usepackage[pdftitle={Assignement 2},  % apparition ds les propriétés du doc
            pdfsubject={Languages and translators},
	    colorlinks=false,
	    linkcolor=webdarkblue, 
	    filecolor=webblue, 
	    urlcolor=webdarkblue,
	    citecolor=webgreen]{hyperref}     % pour l'utilisation des liens http,...

% Police
   \renewcommand\familydefault{ptm}        % famille normale: Times ptm
   %\renewcommand\rmdefault{phv}            % famille à utiliser pour du Roman (phv)
   %\renewcommand\sfdefault{phv}            % famille à utiliser pour du Sans Serif

% L'interligne
   % \onehalfspacing % un et demi (= \setstrech{1.5} ou = \renewcommand{\baselinestretch}{1.5})
   \renewcommand{\baselinestretch}{1.5}

% En-tete
    \lhead{\texttt{LINGI2132} - Assignement 2 - Languages and translators }        \chead{}        \rhead{Baufays - Colmonts}
    %\renewcommand{\headrulewidth}{0.5pt}     % pour l'épaisseur de la ligne

% Bas de page
    \renewcommand{\footrulewidth}{0.5pt}       % pour l'épaisseur de la ligne
    \lfoot{Partie \rightmark}        \cfoot{}        \rfoot{Page \thepage~$-$~2}

% TOC jusqu'au subsection
\setcounter{tocdepth}{2} % Dans la table des matieres
\setcounter{secnumdepth}{2} % Avec un numero.


\begin{document}
\input{title.tex}
\pagenumbering{arabic} % on triche avec la numérotation des pages :)
\linespread{0.30}
\section{Functionality/specificities of our DSL}
Even if our DSL is able to solve any problem, we still added some classes allowing to solve easily some well-known problems. Thus, in the package "dsl.problems", you can find the class Coloring which helps to solve the problem of coloring a set spaces without having the same color side by side. The Knapsack class has also been added, it models the optimisation problem of filling a bag with maximal utility. Finally, the NQueens class proposes to place N Queens on Chess board where no Queen is able to move to another in a single step, following the chess rules.
\subsection{Semantic}
\subsubsection{Declaration of a variable}
Name -> (Range)\newline
Example : "node0" -> (0 to 2) : The variable name will be "node0" and the value that are possible for this variable are {0,1,2}.\newline
Shorthand :
\begin{itemize}
\item Name : declare a variable of name Name and a range from 0 to 1
\item Name to Value : declare a variable of name Name and a range from 0 to Value
\end{itemize}
\subsubsection{Declaration of a constraint}
"SumDsl [>==|<==|equal|dif] SumDsl [Sum|RangeVal]"\newline
We also included implicit conversions from IntVar to Sum, from IntVar to SumDSL, from int ro SumDSL, from Sum to SumDSL and from int to Sum which facilitates constraints declarations.
\subsubsection{Declaration of a constraint array}
"ArrayConstraint [===,!==] ArrayConstraint"\newline
An ArrayConstraint needs a sum set via a closure.
\subsubsection{Sum of variables}
"S(Range, Pas default =1, implicit param Name)"\newline
Returns a sum of variables contained in the range incremented by step "Pas".
Since each variable must have a unique name, we replace each occurence of \% by each value in the range.
\subsubsection{Declaration of a problem}
First, you need to get the SolverDSL object and initialize it (reset it) to be sure it doesn't contain any old constraint from a previous problem. Then, you can add as many variables and constraints as you want easily, as it has been explained above.\\
\section{Documented Model and Advanced DSL Construct}
This part of the report will describe all the classes and objects implemented for this assignment. The DSL code explained is completely contained in the DSL package of the scala project.
\subsection{SolverDSL}
First, we created a class and companion object called SolverDSL which will be the core of our DSL. It contains a solver and two sets. These sets are respectively used to store all the constraints and variables of a specific problem. In the objective of accessing a variable directly with its name, we also used Map with the variable name as key and this variable as value. The "getItem" method is able to return a specific variable. The parameter of these method is implicit because all variables have the same prefix as name. Thus, using this kind of parameter, a user of our DSL can access a variable by using  only its unique number. Defining our SolverDSL as an object, it's easy to access it from other objects. This implementation choice came from the fact that, in RangeVal and Constraint objects, it was really comfortable to add directly constraints and variables to the Solver DSL sets without having to declare an adding method in a class solving a new problem. We rewrote solve and solvewith methods to adapt them to the specificity of our DSL. Indeed, since we don't add directly constraints and variables to the solver, we have to do it before the first call to solve or solveWith. To be sure it's the first call, we decided to use a boolean isAdded to perform the check. We split the process of adding complex constraints (with operators) by using a list. Constraints are added and removed from this list according to constraint construction.
\subsection{RangeVal}
RangeVal is class extending IntVar which allows to represents a problem variables. The companion object contains several implicit conversions whose a very interesting one starting from String to RangeVal. Since SolerDSL is an object, it's possible to know if that String is a variable name already declared in the DSL and , if appropriate, return this variable; otherwise, create a new RangeVal with this name.\\
\subsection{Constraint}
Constraint is a class representing problem specific constraints. Between them, with our implementation, it's possible to use operators AND (\&) and OR (|).\\
\subsection{ArrayConstraint}
ArrayConstraint is class which allows to create constraints f
First, we created a clasor a set of variables. With this construct, the construction of constraints for a problem can be very scalable. In our tests, it was very useful when two variables couldn't have the same value in a small set of possibilities. Since we don't know in advance the sum number that will compose the ArrayConstraint, we used a closure construct of sums.
\subsection{S}
The construct S is able to compute the sum of several variables.
\subsection{SumDsl}
SumDSL object encapsulate Sum and allows us to use all possible operators in order to create constraints. Thus, you can find the following operators : equal, not equal, larger or equal, smaller or equal. We also created a companion object containing several implicit conversions.
\section{Conclusion} 
Our DSL already allows several abstractions and we didn't modified the solver. With our implementation, it's really easy to solve new problems. We facilitated access to variables and constraints, a lot of implicit conversions so a user can add constraints to variables that are not directly linked. Besides problems that are already defined, we added some more test like Sudoku and MagicSquare to convince you that we didn't work in a way to solve only problems given in the assignment. If we had more time, we wanted to add implementation around booleanVar and work even more on helping user to write easily code or sentences that could be understood by the DSL as constraints and variables declaration. 

\end{document}
