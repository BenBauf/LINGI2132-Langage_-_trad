\section{Lexical Analysis}      
\subsection{Give 5 different strings belonging to the language described by this regExp.}
\begin{spacing}{0.1}
\begin{enumerate}
\item aab
\item bbaa
\item ababab
\item aaaaaaa
\item bbbbbbbbbbbbbbbbbbbbbbbbbbbbbbbbab
\end{enumerate}
\end{spacing}
\subsection{Construct the NFA from this regExp (Thompson Construction}
\begin{tikzpicture}[>=stealth',shorten >=1pt,auto,node distance=2cm]
  \node[initial,state] (S)      {$0$};
  \node[state]         (one) [right of=S]  {$1$};
  \node[state]         (two) [above right of=one]  {$2$};
  \node[state]         (three) [right of=two]  {$3$};
  \node[state]         (four) [below right of=one]  {$4$};
  \node[state]         (five) [right of=four]  {$5$};
  \node[state]         (six) [below right of=three]  {$6$};
  \node[state]         (seven) [right of=six]  {$7$};
  \node[state]         (eight) [below of=six, node distance=3.0cm]  {$8$};
  \node[state]         (nine) [above right of=eight]  {$9$};
  \node[state]         (ten) [right of=nine]  {$10$};
  \node[state]         (eleven) [below right of=eight]  {$11$};
  \node[state]         (twelve) [right of=eleven]  {$12$};  
  \node[state, accepting]         (thirteen) [below right of=ten]  {$13$};
  \path[->]          (S)  edge node {$\varepsilon$} (one);
  \path[->]          (one)  edge node {$\varepsilon$} (two);
  \path[->]          (one)  edge node [below left] {$\varepsilon$} (four);
  \path[->]          (two)  edge node {$a$} (three);
  \path[->]          (four)  edge node {$b$} (five);
  \path[->]          (three)  edge node [below left] {$\varepsilon$} (six);
  \path[->]          (five)  edge node {$\varepsilon$} (six);
  \path[->]          (six)  edge node {$\varepsilon$} (seven);
  \path[->]          (six)  edge node {$\varepsilon$} (seven);
  \path[->]          (six)  edge node [above]{$\varepsilon$} (one);
  \path[->]          (S)  edge [bend left=60] node {$\varepsilon$} (seven);
  \path[->]          (seven)  edge [bend right] node [above left]{$a$} (eight);
  \path[->]          (eight)  edge node {$\varepsilon$} (nine);
  \path[->]          (eight)  edge node {$\varepsilon$} (eleven);
  \path[->]          (nine)  edge node {$a$} (ten);
  \path[->]          (eleven)  edge node {$b$} (twelve);
  \path[->]          (ten)  edge node {$\varepsilon$} (thirteen);
  \path[->]          (twelve)  edge node {$\varepsilon$} (thirteen);
 
\end{tikzpicture}


\subsection{Transform the NFA into a DFA (justify important steps, $\varepsilon$-closures).}

Steps from NFA to DFA : \\
\begin{spacing}{0.3}

\begin{itemize}
\item $s_{0} = \epsilon-closure(\{0\}) = \{0,1,2,4,7\}$ 
\item $m(s_{0},a)=s_{1}$, where $s_{1} = \epsilon-closure(\{3,8\}) = \{1,2,3,4,6,7,9,11\}$
\item $m(s_{0},b)=s_{2}$, where $s_{2} = \epsilon-closure(\{5\}) = \{1,2,4,5,6,7\}$
\item $m(s_{1},a)=s_{3}$, where $s_{3} = \epsilon-closure(\{3,8,10\}) = \{1,2,3,4,6,7,8,9,11,13\}$
\item $m(s_{1},b)=s_{4}$, where $s_{4} = \epsilon-closure(\{5,12\}) = \{1,2,4,5,6,7,12,13\}$
\item $m(s_{2},a)=s_{1}$
\item $m(s_{2},b)=s_{2}$
\item $m(s_{3},a)=s_{3}$
\item $m(s_{3},b)=s_{4}$
\item $m(s_{4},a)=s_{1}$
\item $m(s_{4},b)=s_{2}$
\end{itemize}
\end{spacing}

Since $s_{3}$ and $s_{4}$ contain state $13$ which was final in the NFA, they are both final state too.
\begin{center}
\scalebox{0.6}{
\begin{tikzpicture}[>=stealth',shorten >=1pt,auto,node distance=5cm, align=center]
  \node[initial,state] (S)      {$0$ \\ $\{0,1,2,4,7\}$};
  \node[state]         (one) [above right of=S]  {$1$ \\ $\{1,2,3,4,$ \\ $6,7,9,11\}$} ;
  \node[state]         (two) [below right of=S]  {$2$ \\ $\{1,2,4,5,6,7\}$ };
  \node[state, accepting]         (three) [above right of=one]  {$3$ \\ $ \{1,2,3,4,6,$ \\ $7,8,9,10,$ \\ $11,13\}$};
  \node[state, accepting]         (four) [below right of=one]  {$4$ \\ $\{1,2,4,5,6, $\\$7,12,13\}$};  
  \path[->]          (S)  edge node {$a$} (one);
  \path[->]          (S)  edge node {$b$} (two);  
  \path[->]          (two)  edge node {$a$} (one);
  \path[->]          (two)  edge [loop below] node {$b$} (two);
  \path[->]          (one)  edge node {$a$} (three);
  \path[->]          (one)  edge [bend left] node {$b$} (four);
  \path[->]          (three)  edge [loop above] node {$a$} (three);
  \path[->]          (three)  edge node {$b$} (four);
  \path[->]          (four)  edge [bend left] node {$a$} (one);
  \path[->]          (four)  edge node {$b$} (two);  
\end{tikzpicture}
}
\end{center}

\subsection{Minimize the DFA (Hopcroft Algorithm).}
The DFA could be divided in four partitions :
\begin{spacing}{0.3}
\begin{itemize}
\item Two partitions for final states
\item A partition which contains previous states $\{0,2\}$ because :
	\begin{itemize}
	\item $m(0,a) = 1$
	\item $m(2,a) = 1$	
	\item $m(0,b) = 2$
	\item $m(2,b) = 2$	
	\end{itemize}
\item A partition which contains previous state $\{1\}$ :
	\begin{itemize}
	\item $m(1,a) = 3$
	\item $m(1,b) = 4$	
	\end{itemize}
\end{itemize}
\end{spacing}
\begin{center}
\scalebox{0.9}{
\begin{tikzpicture}[>=stealth',shorten >=1pt,auto,node distance=2cm, align=center]
  \node[initial,state] (S)      {$0$};
  \node[state]         (one) [right of=S]  {$1$} ;
  \node[state, accepting]         (two) [above right of=one]  {$2$};
  \node[state, accepting]         (three) [below right of=one]  {$3$};  
  \path[->]          (S)  edge node {$a$} (one);
  \path[->]          (S)  edge [loop below] node {$b$} (S);
  \path[->]          (one)  edge node {$a$} (two);  
  \path[->]          (two)  edge [loop above] node {$a$} (two);
  \path[->]          (one)  edge node {$b$} (three);
  \path[->]          (two)  edge node {$b$} (three);
  \path[->]          (three)  edge node {$b$} (S);

\end{tikzpicture}
}
\end{center}


